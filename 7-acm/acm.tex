\documentclass{acm_proc_article-sp}
\usepackage{mathtools}
\linespread{0.95}

\begin{document}
\title{A Brief Guide to Skateboarding on a Quarter Pipe}

\numberofauthors{1}
\author{
\alignauthor
    Colby Goettel\\
    \affaddr{Brigham Young University}\\
    \affaddr{1 N University Hill}\\
    \affaddr{Provo, UT 84602}\\
    \email{colby.goettel@gmail.com}
}
\date{\today}

\maketitle
\begin{abstract}
    The main focus of this paper is to present skateboarding in a technical way. This presentation calls for a more technical typesetting; hence, the Association for Computing Machinery (ACM) document format is being used. Additionally, this allows for a two column format to be utilized without looking out of place.
    
    This paper will focus on being technical. A major reason for doing so is to use a vertical ellipses.
    
    Finally, less skateboarding vernacular will be used and things will be broken down to the point of near-boredom in an effort to mimic existing technical papers. That said, the writing should not be too dry. Hopefully.
    
    Word count: 332
\end{abstract}

\category{A.2}{General and reference}{Cross-computing tools and techniques}[Empirical studies]
\category{B.8}{Hardware}{Hardware test}[Board- and system-level test]
\category{I.2}{Human-centered computing}{Interaction design}[Interaction design process and methods: User interface design]

\terms{skateboarding, board fundamentals}

\keywords{skateboarding, quarter pipe, style}

\section{Brief Cost-Benefit Analysis}
Learning the basics of skateboarding takes time and dedication. It can take months, even years, to learn the basics of skateboarding. There is neither coach nor trainer, simply the skateboarding against himself. Because of this fact and the intrinsic, dangerous nature of skateboarding, the potential for injury is greatly increased. In fact, having a session at the skatepark without minimal pain is rare.

The pain induced by skateboarding is inherent in the sport: it is part of the very nature and territory of skateboarding. Those wishing to pursue this lifestyle recognize and accept this risk in their lives.

\section{Learning to ride a quarter pipe}
This section details the basic steps necessary in learning how to ride a quarter pipe, one of the fundamental skateboarding obstacles.

\subsection{Pivoting}
The first step is to learn how to pivot, first on flat ground, and then on a ramp. Stand on the skateboard, do not push off, and proceed as follows:
\vspace{-1em}\begin{enumerate}
    \itemsep0em
    \item Manual slightly.\footnote{A trick where the skateboarder balances on the rear wheels.}
    \item Quickly pivot to the right.
    \item Place front trucks on ground.
    \item Manual slightly.
    \item Quickly pivot to the left.
    \item Place front trucks on ground.
    \item Repeat this process.\footnote{Note that the skateboard will gain forward momentum during this process.}
    
    \vspace{-0.25em}\hspace{6em}\vdots
    \item This is considered done when it can be performed without falling off the skateboard.
\end{enumerate}

\vspace{-1em}Next, approach a medium-sized ramp with enough speed to only go halfway up, pivot 90$^\circ$, pivot an additional 90$^\circ$, and ride back down. Once comfortable, try to pivot the full 180$^\circ$ without touching down partway through. Practice this maneuver until it becomes second nature.

\subsection{Reapplying knowledge}
Once a skateboarder can pivot, he can then transfer this knowledge to a quarter pipe. Approach the quarter pipe in the same fashion as when pivoting on a ramp. Approach and bend the knees right before the transition. At the apex, pivot the full 180$^\circ$, maintain balance, and ride back down.

\section{Conclusion}
Skateboarding can greatly improve one's well-being. Learning to commit in order to accomplish established goals brings with it a sense of satisfaction only attained through rigorous trial.

\balancecolumns

\end{document}
